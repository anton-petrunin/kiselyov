%расстояние ... от ...

%ёфикация

%восставить/восстановить

%окружность/круг центра --> окружность с центром в

%коли --> если

%избавится от d.

добавить знак прямого угла.

обозначать углы только греческими буквами

%перпендикуляр проходящий через середину отрезка --> срединный перпендикуляр

%единообразное обозначение расстояний на чертежах.

сходственные --> соответственные

%: --> дробь.

преобразование подобия --- гомотетия или переспективное подобие/перспективно подобные???

%убрать «суть»

%= и \approx

%замечания о радианах как основной единице

%приведение к нелепости

%l --> \ell

нумерация иногда 1), иногда 1.

действительные --> вещественные ?

%многоугольник --> $n$-угольник

порядок букв в подобных/равных треугольниках.

вогнутые/ невыпуклые многоугольники 

\medskip до \so 1) 1. ...

общий подход к названиям теорем (Фалеса, Пифагора...)

ось симметрии --- штрих-пунктир

\bm в жирном шрифрте

численная величина, численная мера= длина

измеренные одной единицей

Па\-ра\-лле\-ло\-грамм

дуга окружности

теперь ???